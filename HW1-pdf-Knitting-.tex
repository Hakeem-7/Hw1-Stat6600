\documentclass[]{article}
\usepackage{lmodern}
\usepackage{amssymb,amsmath}
\usepackage{ifxetex,ifluatex}
\usepackage{fixltx2e} % provides \textsubscript
\ifnum 0\ifxetex 1\fi\ifluatex 1\fi=0 % if pdftex
  \usepackage[T1]{fontenc}
  \usepackage[utf8]{inputenc}
\else % if luatex or xelatex
  \ifxetex
    \usepackage{mathspec}
  \else
    \usepackage{fontspec}
  \fi
  \defaultfontfeatures{Ligatures=TeX,Scale=MatchLowercase}
\fi
% use upquote if available, for straight quotes in verbatim environments
\IfFileExists{upquote.sty}{\usepackage{upquote}}{}
% use microtype if available
\IfFileExists{microtype.sty}{%
\usepackage{microtype}
\UseMicrotypeSet[protrusion]{basicmath} % disable protrusion for tt fonts
}{}
\usepackage[margin=1in]{geometry}
\usepackage{hyperref}
\hypersetup{unicode=true,
            pdftitle={HW 1 (STAT 6600)},
            pdfauthor={Akeem Ajede},
            pdfborder={0 0 0},
            breaklinks=true}
\urlstyle{same}  % don't use monospace font for urls
\usepackage{color}
\usepackage{fancyvrb}
\newcommand{\VerbBar}{|}
\newcommand{\VERB}{\Verb[commandchars=\\\{\}]}
\DefineVerbatimEnvironment{Highlighting}{Verbatim}{commandchars=\\\{\}}
% Add ',fontsize=\small' for more characters per line
\usepackage{framed}
\definecolor{shadecolor}{RGB}{248,248,248}
\newenvironment{Shaded}{\begin{snugshade}}{\end{snugshade}}
\newcommand{\AlertTok}[1]{\textcolor[rgb]{0.94,0.16,0.16}{#1}}
\newcommand{\AnnotationTok}[1]{\textcolor[rgb]{0.56,0.35,0.01}{\textbf{\textit{#1}}}}
\newcommand{\AttributeTok}[1]{\textcolor[rgb]{0.77,0.63,0.00}{#1}}
\newcommand{\BaseNTok}[1]{\textcolor[rgb]{0.00,0.00,0.81}{#1}}
\newcommand{\BuiltInTok}[1]{#1}
\newcommand{\CharTok}[1]{\textcolor[rgb]{0.31,0.60,0.02}{#1}}
\newcommand{\CommentTok}[1]{\textcolor[rgb]{0.56,0.35,0.01}{\textit{#1}}}
\newcommand{\CommentVarTok}[1]{\textcolor[rgb]{0.56,0.35,0.01}{\textbf{\textit{#1}}}}
\newcommand{\ConstantTok}[1]{\textcolor[rgb]{0.00,0.00,0.00}{#1}}
\newcommand{\ControlFlowTok}[1]{\textcolor[rgb]{0.13,0.29,0.53}{\textbf{#1}}}
\newcommand{\DataTypeTok}[1]{\textcolor[rgb]{0.13,0.29,0.53}{#1}}
\newcommand{\DecValTok}[1]{\textcolor[rgb]{0.00,0.00,0.81}{#1}}
\newcommand{\DocumentationTok}[1]{\textcolor[rgb]{0.56,0.35,0.01}{\textbf{\textit{#1}}}}
\newcommand{\ErrorTok}[1]{\textcolor[rgb]{0.64,0.00,0.00}{\textbf{#1}}}
\newcommand{\ExtensionTok}[1]{#1}
\newcommand{\FloatTok}[1]{\textcolor[rgb]{0.00,0.00,0.81}{#1}}
\newcommand{\FunctionTok}[1]{\textcolor[rgb]{0.00,0.00,0.00}{#1}}
\newcommand{\ImportTok}[1]{#1}
\newcommand{\InformationTok}[1]{\textcolor[rgb]{0.56,0.35,0.01}{\textbf{\textit{#1}}}}
\newcommand{\KeywordTok}[1]{\textcolor[rgb]{0.13,0.29,0.53}{\textbf{#1}}}
\newcommand{\NormalTok}[1]{#1}
\newcommand{\OperatorTok}[1]{\textcolor[rgb]{0.81,0.36,0.00}{\textbf{#1}}}
\newcommand{\OtherTok}[1]{\textcolor[rgb]{0.56,0.35,0.01}{#1}}
\newcommand{\PreprocessorTok}[1]{\textcolor[rgb]{0.56,0.35,0.01}{\textit{#1}}}
\newcommand{\RegionMarkerTok}[1]{#1}
\newcommand{\SpecialCharTok}[1]{\textcolor[rgb]{0.00,0.00,0.00}{#1}}
\newcommand{\SpecialStringTok}[1]{\textcolor[rgb]{0.31,0.60,0.02}{#1}}
\newcommand{\StringTok}[1]{\textcolor[rgb]{0.31,0.60,0.02}{#1}}
\newcommand{\VariableTok}[1]{\textcolor[rgb]{0.00,0.00,0.00}{#1}}
\newcommand{\VerbatimStringTok}[1]{\textcolor[rgb]{0.31,0.60,0.02}{#1}}
\newcommand{\WarningTok}[1]{\textcolor[rgb]{0.56,0.35,0.01}{\textbf{\textit{#1}}}}
\usepackage{graphicx,grffile}
\makeatletter
\def\maxwidth{\ifdim\Gin@nat@width>\linewidth\linewidth\else\Gin@nat@width\fi}
\def\maxheight{\ifdim\Gin@nat@height>\textheight\textheight\else\Gin@nat@height\fi}
\makeatother
% Scale images if necessary, so that they will not overflow the page
% margins by default, and it is still possible to overwrite the defaults
% using explicit options in \includegraphics[width, height, ...]{}
\setkeys{Gin}{width=\maxwidth,height=\maxheight,keepaspectratio}
\IfFileExists{parskip.sty}{%
\usepackage{parskip}
}{% else
\setlength{\parindent}{0pt}
\setlength{\parskip}{6pt plus 2pt minus 1pt}
}
\setlength{\emergencystretch}{3em}  % prevent overfull lines
\providecommand{\tightlist}{%
  \setlength{\itemsep}{0pt}\setlength{\parskip}{0pt}}
\setcounter{secnumdepth}{0}
% Redefines (sub)paragraphs to behave more like sections
\ifx\paragraph\undefined\else
\let\oldparagraph\paragraph
\renewcommand{\paragraph}[1]{\oldparagraph{#1}\mbox{}}
\fi
\ifx\subparagraph\undefined\else
\let\oldsubparagraph\subparagraph
\renewcommand{\subparagraph}[1]{\oldsubparagraph{#1}\mbox{}}
\fi

%%% Use protect on footnotes to avoid problems with footnotes in titles
\let\rmarkdownfootnote\footnote%
\def\footnote{\protect\rmarkdownfootnote}

%%% Change title format to be more compact
\usepackage{titling}

% Create subtitle command for use in maketitle
\providecommand{\subtitle}[1]{
  \posttitle{
    \begin{center}\large#1\end{center}
    }
}

\setlength{\droptitle}{-2em}

  \title{HW 1 (STAT 6600)}
    \pretitle{\vspace{\droptitle}\centering\huge}
  \posttitle{\par}
    \author{Akeem Ajede}
    \preauthor{\centering\large\emph}
  \postauthor{\par}
      \predate{\centering\large\emph}
  \postdate{\par}
    \date{9/10/2019}


\begin{document}
\maketitle

\hypertarget{stat-6600-assignment-1}{%
\section{STAT 6600: Assignment 1}\label{stat-6600-assignment-1}}

\hypertarget{due-september-10-2019}{%
\subsection{Due: September 10, 2019}\label{due-september-10-2019}}

\hypertarget{akeem-ajede}{%
\subsection{Akeem Ajede}\label{akeem-ajede}}

\medskip

\hypertarget{question-1a}{%
\subsubsection{Question 1(a)}\label{question-1a}}

Find the sample mean \(\bar{x}\) and standard deviation \(s\). Then
compute the intervals {[}\(\bar{x}-s, \bar{x}+s\){]},
{[}\(\bar{x}-2s, \bar{x}+2s\){]} and {[}\(\bar{x}-3s, \bar{x}+3s\){]}
and compute the percentage of sample values that fall within each
interval.

\hypertarget{solution}{%
\subsubsection{Solution}\label{solution}}

\(\mathbf{Approach}\): Stem and Leaf

The sample mean is equivalent to the 50th percentile \(P_{50}\), which
was computed as delineated below:

\((n+1)p=(97)\frac{1}{2}=48.5 = 48+\frac{1}{2}\),

The sample mean is located between the 48th and 49th integer.

\(\hat{\pi}_{\frac{1}{2}}=(1-\frac{1}{2})x_{48}+(\frac{1}{2})x_{49}=(\frac{1}{2})16.6+(\frac{1}{2})16.7=16.65\)

From sigma rule, \(95\%\) of the data lies within \(2s\) from \(\mu\).
Hence, \(5\%\) of the data are beyond \(2s\), which is approximately
\(96*5\%=4.8,\) (i.e., about 4 or 5 observations). From the 16th stem, 4
and 3 stems from the right and left tail, respectively will keep 4
observations beyond 2 standard deviations from the mean. By computing
the average, \(\frac{3+4}{2}=3.5=2\sigma,\) Thus, \(\sigma=1.75\)

\(\mathbf{Intervals:}\)

\([\bar{x} - s, \bar{x} + s] = [16.7 - 1.75, 16.7 + 1.75] = [15.0, 18.5]\)
which contains 65 values \(\approx 67.7\%\) of the data.

\([\bar{x} - 2s, \bar{x} + 2s] = [16.7 - 2(1.75), 16.7 + 2(1.75)] = [13.2, 20.2]\)
which contains 91 values \(\approx 94.7\%\) of the data.

\([\bar{x} - 3s, \bar{x} + 3s] = [16.7 - 3(1.75), 16.7 + 3(1.75)] = [11.45, 21.95]\)
which contains 95 values \(\approx 98.9\%\) of the data.

\(\mathbf{Note:}\) The sample percentages computed approximately
corresponds to the established proportions of a standard normal
distribution.

\medskip

\hypertarget{question-1b}{%
\subsubsection{Question 1(b)}\label{question-1b}}

Find the minimum \(x_{(1)}\) and the maximum \(x_{(96)}\) order
statistics, the 25th, 50th and 75th percentiles (also known as the 1st
quartile \(Q1\), the median \(\tilde{x}\) and the third quartile \(Q3\),
respectively. Use the stem-and-leaf plot to do so. Also,find the
interquartile range \(IQR = Q3 - Q1\).

\hypertarget{solution-1}{%
\subsubsection{Solution}\label{solution-1}}

\(x_{(1)}=11.9\) and \(x_{(96)}=22.1\),

\(P_{25}=Q_1\)

\(p=\frac{25}{100}=\frac{1}{4}\),
\((n+1)p=(97)\frac{1}{4}=22.25 = 22+\frac{1}{4}\),

\(\hat{\pi}_{\frac{1}{4}}=(1-\frac{1}{4})x_{22}+(\frac{1}{4})x_{23}=(\frac{3}{4})15.3+(\frac{1}{4})15.4=15.3\)

\(P_{50}\) \(P_{50} = \bar{x}=16.65\) (Previously computed)

\(P_{75}=Q_3\)

\(p=\frac{75}{100}=\frac{3}{4}\),
\((n+1)p=(97)\frac{3}{4}=72.75 = 72+\frac{3}{4}\),

\(\hat{\pi}_{\frac{3}{4}}=(1-\frac{3}{4})x_{72}+(\frac{3}{4})x_{73}=(\frac{3}{4})17.8+(\frac{1}{4})17.8=17.8\),

\(IQR = Q_3-Q_1=17.8-15.3=2.5\)

\hypertarget{question-1c}{%
\subsubsection{Question 1(c)}\label{question-1c}}

Find the 5th and 95th percentiles \(p_5\) and \(p_{95}\)

\hypertarget{solution-2}{%
\subsubsection{Solution}\label{solution-2}}

\(\mathbf{P_5}\)

\(p=\frac{5}{100}=\frac{1}{20}\),
\((n+1)p=(97)\frac{1}{20}=22.25=22+\frac{1}{4}\)

\(\hat{\pi}_{\frac{1}{20}}=(1-\frac{1}{20})x_{4}+(\frac{1}{20})x_{5}=(\frac{19}{20})13.7+(\frac{1}{20})14.1=\mathbf{13.7}\)

\(\mathbf{P_{95}}\)

\(p=\frac{95}{100}=\frac{19}{20}\),
\((n+1)p=(97)\frac{19}{20}=92.15=92+\frac{3}{20}\),

\(\hat{\pi}_{\frac{19}{20}}=(1-\frac{19}{20})x_{92}+(\frac{19}{20})x_{93}=(\frac{1}{20})19.8+(\frac{19}{20})20.2=\mathbf{20.18}\)

\hypertarget{question-1d}{%
\subsubsection{Question 1(d)}\label{question-1d}}

Assuming that the data are from a normal population, and an expression
for the maximum likelihood estimator of the CDF,

\[F_X(x)=\frac{1}{\sqrt{2\pi}\sigma}\int_{-\infty}^xexp\Big\{ -\frac{(t-\mu)^2}{2\sigma^2}\Big\}dt.\]

\hypertarget{solution-3}{%
\subsubsection{Solution}\label{solution-3}}

Using the invariant property of MLEs, an estimate of the CDF is given as
\[F_X(x|\hat\theta)=\frac{1}{\sqrt{2\pi}\hat\sigma}\int_{-\infty}^xexp\Big\{ -\frac{(t-\hat\mu)^2}{2\hat\sigma^2}\Big\}dt,\]

where, \(\hat\mu=\bar X=16.7\), and
\(\hat\sigma^2=\frac{\sum_{i=1}^{96}(x_i-\bar x)^2}{n}=\mathbf{3.396}\)

\hypertarget{question-1e}{%
\subsubsection{Question 1(e)}\label{question-1e}}

Use R to generate a histogram of the sample distribution with the
``theoretical'' pdf overlaid. \vskip

\hypertarget{solution-4}{%
\subsubsection{Solution}\label{solution-4}}

\begin{Shaded}
\begin{Highlighting}[]
\NormalTok{Raw =}\StringTok{ }\KeywordTok{read.table}\NormalTok{(}\StringTok{"Prob1InjectorPumps.txt"}\NormalTok{, }\DataTypeTok{header =}\NormalTok{ F)}
\NormalTok{y =}\StringTok{ }\NormalTok{Raw[}\KeywordTok{order}\NormalTok{(Raw}\OperatorTok{$}\NormalTok{V1),]}
\KeywordTok{hist}\NormalTok{(y, }\DataTypeTok{prob =}\NormalTok{ T, }\DataTypeTok{xlab =} \StringTok{"Plunger Relative Diameter (microns)"}\NormalTok{, }\DataTypeTok{ylab =} \StringTok{"count"}\NormalTok{, }\DataTypeTok{col =} \StringTok{"green"}\NormalTok{, }\DataTypeTok{main =} \StringTok{"Histogram of Plunger Relative Diameter"}\NormalTok{)}
\NormalTok{m <-}\KeywordTok{mean}\NormalTok{(y);std <-}\KeywordTok{sqrt}\NormalTok{(}\KeywordTok{var}\NormalTok{(y))}
\KeywordTok{curve}\NormalTok{(}\KeywordTok{dnorm}\NormalTok{(x, }\DataTypeTok{mean=}\NormalTok{m, }\DataTypeTok{sd=}\NormalTok{std), }\DataTypeTok{col=} \DecValTok{2}\NormalTok{, }\DataTypeTok{lwd=}\DecValTok{2}\NormalTok{, }\DataTypeTok{add=}\OtherTok{TRUE}\NormalTok{)}
\end{Highlighting}
\end{Shaded}

\includegraphics{HW1-pdf-Knitting-_files/figure-latex/unnamed-chunk-1-1.pdf}

\medskip

\hypertarget{question-1f}{%
\subsubsection{Question 1(f)}\label{question-1f}}

Use R to generate the empirical CDF with the ``theoretical'' CDF
overlaid.

\hypertarget{solution-5}{%
\subsubsection{Solution}\label{solution-5}}

\begin{Shaded}
\begin{Highlighting}[]
\KeywordTok{plot}\NormalTok{(}\KeywordTok{ecdf}\NormalTok{(y), }\DataTypeTok{main =} \StringTok{"ECDF of Plunger Relative Diameter (microns)"}\NormalTok{)}
\KeywordTok{curve}\NormalTok{(}\KeywordTok{pnorm}\NormalTok{(x, m, std), }\DataTypeTok{col=} \DecValTok{2}\NormalTok{, }\DataTypeTok{lwd=}\DecValTok{2}\NormalTok{, }\DataTypeTok{add=}\OtherTok{TRUE}\NormalTok{)}
\end{Highlighting}
\end{Shaded}

\includegraphics{HW1-pdf-Knitting-_files/figure-latex/unnamed-chunk-2-1.pdf}

\hypertarget{question-1g}{%
\subsubsection{Question 1(g)}\label{question-1g}}

Use R to produce a normal QQ plot.

\hypertarget{solution-6}{%
\subsubsection{Solution}\label{solution-6}}

\begin{Shaded}
\begin{Highlighting}[]
\KeywordTok{qqnorm}\NormalTok{(y)}
\KeywordTok{qqline}\NormalTok{(y, }\DataTypeTok{col =} \StringTok{"black"}\NormalTok{, }\DataTypeTok{lwd =} \DecValTok{2}\NormalTok{)}
\end{Highlighting}
\end{Shaded}

\includegraphics{HW1-pdf-Knitting-_files/figure-latex/unnamed-chunk-3-1.pdf}

\vskip

\hypertarget{question-1h}{%
\subsubsection{Question 1(h)}\label{question-1h}}

Does the data look like it came from a normal population?l

\hypertarget{solution-7}{%
\subsubsection{Solution}\label{solution-7}}

From the normality plot (i.e., qq - plot), the data form an
approximately straight line along the normality line. Hence, it is safe
to conclude that the data came from a normal population.

\vskip

\hypertarget{question-1i}{%
\subsubsection{Question 1(i)}\label{question-1i}}

Recall that since
\(\bar{X} \approx N(\mu_{\bar{X}},\sigma^2_{\bar{X}}) = N(\mu, \frac{\sigma^2}{n}),\)
the \(SE(\hat{\mu})=\frac{s}{\sqrt{n}}\). Compute a \(90\%\) confidence
interval on the true mean plunger force.

\hypertarget{solution-8}{%
\subsubsection{Solution}\label{solution-8}}

\(\sum_{i=1}^{n}x_i = 1603.8\),
\(\bar{x}=\frac{1}{n}\sum_{i=1}^{n}x_i\),
\(\bar{x}=\frac{1}{96}*1603.8=16.706\), \(\sum_{i=1}^nx_i^2=27119.48\),
\(S = \sqrt{\frac{\sum x_i^2-(\sum x_i)^2/n}{n}}\), \(S = 1.843\),
\(t_\frac{\alpha}{2},_{n-2}= t_{0.05,94} = 0.397\),
\(S_n = \sqrt{\frac{\sum x_i^2-(\sum x_i)^2/n}{n}}\),
\(\sum_{i=1}^nx_i^2=27119.48\), \(S_n = 1.843\)

Interval: \(\bar{x}\pm t_\frac{\alpha}{2},_{n-2} *\frac{S_n}{\sqrt{n}}\)
= \(16.706 \pm(-0.312)\),

\((16.394\leqslant \bar{x} \leqslant 17.018\)

\vskip

\hypertarget{question-2e}{%
\subsubsection{Question 2(e)}\label{question-2e}}

\begin{Shaded}
\begin{Highlighting}[]
\NormalTok{q2 =}\StringTok{ }\KeywordTok{c}\NormalTok{(}\DecValTok{106}\NormalTok{, }\DecValTok{4972}\NormalTok{, }\DecValTok{7140}\NormalTok{, }\DecValTok{7661}\NormalTok{, }\DecValTok{1776}\NormalTok{, }\DecValTok{2471}\NormalTok{, }\DecValTok{5550}\NormalTok{, }\DecValTok{6959}\NormalTok{, }\DecValTok{3541}\NormalTok{, }\DecValTok{747}\NormalTok{, }\DecValTok{5142}\NormalTok{, }\DecValTok{25691}\NormalTok{, }\DecValTok{11345}\NormalTok{, }\DecValTok{10067}\NormalTok{, }\DecValTok{5434}\NormalTok{)}
\KeywordTok{qqnorm}\NormalTok{(q2, }\DataTypeTok{col =} \DecValTok{2}\NormalTok{)}
\KeywordTok{qqline}\NormalTok{(q2, }\DataTypeTok{col =} \StringTok{"black"}\NormalTok{, }\DataTypeTok{lwd =} \DecValTok{2}\NormalTok{)}
\end{Highlighting}
\end{Shaded}

\includegraphics{HW1-pdf-Knitting-_files/figure-latex/unnamed-chunk-4-1.pdf}

\begin{Shaded}
\begin{Highlighting}[]
\KeywordTok{plot}\NormalTok{(}\KeywordTok{ecdf}\NormalTok{(q2), }\DataTypeTok{main =} \StringTok{"Burn out times from various locations"}\NormalTok{)}
\KeywordTok{curve}\NormalTok{(}\KeywordTok{pexp}\NormalTok{(x,}\FloatTok{0.000152}\NormalTok{), }\DataTypeTok{add=}\OtherTok{TRUE}\NormalTok{, }\DataTypeTok{col=} \DecValTok{2}\NormalTok{)}
\end{Highlighting}
\end{Shaded}

\includegraphics{HW1-pdf-Knitting-_files/figure-latex/unnamed-chunk-5-1.pdf}

From the plot of the empiral CDF, it can be deduced that the sample data
is from an exponential population data.


\end{document}
